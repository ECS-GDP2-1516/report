\chapter{Risk Assessment}

The project''s risk assessment is shown opposite, with a red/amber/green traffic  light scheme supporting the impact scores for ease of reference.

\begin{landscape}
\begin{center}
\begin{tabular}{ | m{2cm} | m{7cm} | m{2cm} | m{1.5cm} | m{1.5cm} | m{7cm} | } 
  \hline
  Risk & Description & Likelihood & Impact & Priority & Mitigation \\
  \hline

  Not able to recognise exercises &
  The possibility that no algorithm can be discovered or created to distinguish exercise from non-exercise &
  0.3 & 0.9 & 0.27 &
  Use Weka, a well known and well tested, tool designed to bulk test multiple existing machine learning algorithms \\
  \hline
 
  Unable to scale algorithm &
  The system we have been given does not allow division or floating point. It may not be possible to reduce an algorithm to this constraints without prohibitively low running time &
  0.5 & 0.7 & 0.35 &
  Use this constraint when deciding an optimum algorithm: where possible, choose methods which require less computation and, specifically, fewer operations relying on complex mathematical operations which require accurate floating point or division. \\
  \hline
 
  Loss of work &
  Technical, physical or human issues may lead to loss of work or data, which will delay progress
  & 0.3 & 0.9 & 0.27 &
  All code kept under Git version control, where this is not under licence it will be backed up to GitHub. Repositories which include non-publishable code will be backed up to a personal git server of one of the team members. All documents written using LaTeX, which will also be kept under version control, or Google Documents. \\
  \hline
  
  Code cannot be compiled &
  Issues with library compatibilities or compiler limitations may cause problems when compiling our code for the device &
  0.1 & 0.8 &
  0.08 &
  Use ARM's provided compiler, which is well supported and relies on common libraries. \\
  \hline
\end{tabular}

\begin{tabular}{ | m{2cm} | m{7cm} | m{2cm} | m{1.5cm} | m{1.5cm} | m{7cm} | } 
  \hline
  Risk & Description & Likelihood & Impact & Priority & Mitigation \\
  \hline

  Code does not fit on device &
  When compiling the code, its size may exceed the maximum of 8KB meaning it will not fit onto the device. This may happen because too many libraries are required or the algorithm itself is too large &
  0.4 & 0.8 & 0.32 &
  Use this constraint when deciding an optimum algorithm: where possible, choose lower sampling frequencies as these require less weights \\
  \hline

  Developer Health &
  Ongoing health issues with one team member may disrupt their ability to work &
  0.8 & 0.6 & 0.48 &
  . \\
  \hline
  
  mBed clock speed &
  The mBed's specification states that it may have an error on its clock speed, making measurements difficult &
  0.1 & 0.1 & 0.01 &
  Test clock speeds under various conditions to see the impact. Also develop on the FPGA which does not have this issue. \\
  \hline
  
  FPGA &
  The FPGA is a complex device - if we are unable to obtain or understand the relevant documentation, it will be difficult to complete the required verilog &
  0.3 & 0.5 & 0.15 &
  Ask ARM for the required documentation and maintain contact in case of issues. \\
  \hline
  
  Ethical Approval Delays &
  Issues receiving ethical approval may delay the date at which initial collection can be done, which would in turn delay the development schedule. &
  0.6 & 0.6 & 0.36 &
  Plan to complete the ethical approval request early in the project. Ensure the development schedule does not rely on the movement data from the very beginning. \\
  \hline
 
\end{tabular}
\end{center}
\end{landscape}
