\namedsection{mBed}{Gupta}

\subsection{Introduction}

mBed is a platform which implements ARM 32-bit processors within micro-controllers which is primarily developed by ARM. They are of the DIP form factor with 40 pins. It features an online SDK allowing for the development of projects online. It permits code to be written online as well as compiled into a binary compatible with the board being used. This binary can then be downloaded removing the pre-requisite of having the ARM toolchain available locally. The manner of uploading to the mBed is relatively straight forward with the presence of the mBed interface. \cite{mbed_website}\todo{Try expanding on this section}

This interface exposes a Mass Storage device to the host computer via a USB connection allowing for binaries to be uploaded in a drag and drop manner. The interface is also connected with the target chip using a JTAG connection allowing for it to program its flash memory. When the reset button is pressed, the interface checks the storage for the newest binary file and, should it not be programmed within the device already, will program the binary provided into the flash memory of the chip. \cite{mbed_website}\todo{Try expanding this section as well}

\subsection{Why the mBed}
\note{Need a better name for this subsection?}

When initially looking at various ways to emulate the sub-threshold version of the ARM Cortex M0, using an existing non sub-threshold version of the ARM Cortex M0 was considered. This involved searching for some form of micro-controller or device which would allow for programming of the processor. This lead to the mBed family which appeared easy to program and use featuring ARM processors. Within the mBed family of micro-controllers, the LPC11U24 model uses an ARM Cortex M0 as its processor.

The requirements for this project require the processor to be emulate the sub-threshold version which operates with a system clock in the range of a few hundred kHz to a few MHz. This would require a clock in the micro-controller which can operate at these frequencies and also be able to be set as the system clock.

The device typically runs at 48MHz using its Internal RC oscillator (IRC) as its system clock. However, it also has the ability to switch its system clock to be sourced from the internal Watchdog oscillator instead. 

This oscillator consists of two parts, an oscillator function which generates an analog clock (\verb|Fclkana|) as well as a divisor (\verb|DIVSEL|) which is used to divide the analog clock to the required output frequency (\verb|wdt_osc_clk|). The output frequency can be calculated using Equation~\ref{eq:wdt_osc_clk}. This allows for the Watchdog oscillator to have a frequency in the range $ 9.4 $ kHz $ \leq \verb|wdt_osc_clk| \leq 2.3 $ MHz. \cite{mbed_datasheet}

\begin{equation}
	\label{eq:wdt_osc_clk}
\end{equation}
\todo{fill in equation wdt\_osc\_clk = Fclkana / (2 * (1 + DIVSEL))}

Some concerns were made realising that the Watchdog oscillator has an error margin of $\pm 40\%$ for the frequency of Fclkana. In a meeting with the client, it was decided to carry forward with the device while paying attention to this margin and analysing the impact of it upon the project.

 

\note{Limitations of mBed, e.g. hardware divide, floating point arithmetic}
\note{Serial communication - autobaud, python script}
\note{I2C with sensor}
\note{Receiving sensor data}
\note{Configuring Watchdog Oscillator for main clock}
\note{Use output pin in while loop due to clkout pin not on the mBed breakout}
\note{I2C pulse and Watchdog setting to optimise sensor sampling rate}
\note{SPI mBed library cannot be compiled because it is bigger than the flash memory}
\note{Bitshifting gives random outputs fun}