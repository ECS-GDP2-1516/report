\namedsection{Space Restrictions \label{sec:space-restrictions}}{Shepherd}

As highlighted in section \ref{sec:cortex-limitations}, the device has very little memory space availiable: 8kB for both program code and data. When the algorithm code was first compiled with the mBed's precompiled libraries, the total size of the binary was 12kB. The first step to solve this was to obtain the source code for the mBed libraries; compiling these directly along with the program code, offers further potential for the compiler to optimise the link between the API and the algorithm, based on the specific usecase.

The method was successful at reducing the total binary size to 8kB. This was a substancial improvement, however it was clear that this would need to be decreased further as an 8kB binary leaves no room for heap or stack space when the code is run.

\subsection{Improving GCC's Optimisations}

While ARM's compiler is very good at optimising code when it compiles C and C++ code, the linker is not able to do so many of the same optimisations across compilation units. It was a logical first step, therefore, to identify the functions and classes that the program code calls within the mBed's API to investigate the benefits of combining their compilation unit.

In many cases, specific API calls were only made once from the algorithm code, such as the call to \verb|ic2_init()| which gets the I2C ready to communicate with the processor at the beginning of the program. Within a library, it is logical to ensure such a function remains an atomic and referenceable item, as it is feasible for a device to use more than one I2C device; as such the call may need to be made multiple times and with varying parameters.

However, as it was only called once in this case, moving \verb|ic2_init()| into the same compilation unit as the function which calls it, replacing the call with the code itself inline. As a result, the compiler is no longer required to produce assembly instructions to copy in arguments and push a new frame onto the stack; this saves not only space, but improves the performance of the code too.

\subsection{Argument Guards}

The mBed API's functions also often came with guards on their arguments; for example, the function \verb|gpio_init()| which is used to initialise the LED pins, accepts a pin as an argument. Before initialising the given pin, it checks that it was not passed the pseudo-pin ``Not Connected''.

\begin{lstlisting}[caption={Argument Guards of gpio init}]
void gpio_init(gpio_t *obj, PinName pin)
{
    if (pin == (PinName)NC) // NC = Not Connected
        return;

    // gpio_init code...
}
\end{lstlisting}

Again, such a check is appropriate for an API as incorrectly passing a null pin would cause an error should the code attempt to initialise it. However, within this project's code, it is known which pins are being passed to this function, namely the LED pins, and not the null \verb|NC| pin. As such it is safe to delete the check, reducing the number of instructions required.

\subsection{Device-Specific Memory Mapping}

The mBed library is designed to be general purpose across multiple platforms. As such the library contains a layer of abstraction to help it interface with the pins and memory addresses for a specific device.

\subsubsection{Ideal Memory Mapping}\label{section:constant-memory-map}

For memory addresses, this abstraction is achieved using macros in device-specific header files, which allows optimisation to can happen at compile time:

\begin{lstlisting}[caption={Memory spaces mapped in LPC11Uxx.h}]
typedef struct {                 /*!< GPIO_GROUP_INT0 Structure */
    __IO uint32_t CTRL;          /*!< GPIO grouped interrupt control register */
    __I  uint32_t RESERVED0[7];
    __IO uint32_t PORT_POL[2];   /*!< Interrupt port 0 polarity register */
    __I  uint32_t RESERVED1[6];
    __IO uint32_t PORT_ENA[2];   /*!< Interrupt port 0/1 enable register */
} LPC_GPIO_GROUP_INTx_Type;

// Peripheral memory map
#define LPC_GPIO_PIN_INT_BASE     (0x4004C000)
#define LPC_GPIO_GROUP_INT0_BASE  (0x4005C000)
#define LPC_GPIO_GROUP_INT1_BASE  (0x40060000)

#define LPC_GPIO_GROUP_INT0 ((LPC_GPIO_GROUP_INTx_Type*) LPC_GPIO_GROUP_INT0_BASE)
#define LPC_GPIO_GROUP_INT1 ((LPC_GPIO_GROUP_INTx_Type*) LPC_GPIO_GROUP_INT1_BASE)
\end{lstlisting}

The above specifies the structure of the GPIO INT memory space, then defines the memory spaces and creates references to these. This allows the developer, and the libraries to use the constant pointers, as illustrated:

\begin{lstlisting}[caption={LPC GPIO GROUP INT0 being used}]
#include ``LPC11Uxx.h'';

void main()
{
    LPC_GPIO_GROUP_INT0_BASE->CTRL = 48;
}
\end{lstlisting}

The use of these definitions allows the compiler to optimise these to constant values, as shown with the assembly below:

\begin{lstlisting}[caption={LPC GPIO GROUP INT0 converted to ASM}]
main:
    str fp, [sp, #-4]! @ Stack Init
    add fp, sp, #0     @ Stack Init
    ldr r3, .L2        @ Value of memory space
    mov r2, #48        @ Using 48
    str r2, [r3]       @ Saving 48 to the memory space
    mov r0, r3
    sub sp, fp, #0
    @ sp needed
    ldr fp, [sp], #4
    bx  lr
.L3:
    .align  2
.L2:
    .word   1074118656 @ This is 0x4005C000
\end{lstlisting}

\subsubsection{Inefficient Mapping}

Unfortunately, a similar technique is not used for the Pin Mappings. Instead, these are stored in an array within a C file which is specific to each target, as shown below. These values are worked upon in two separate files - one common across all devices, and one common to the device's family.

\begin{lstlisting}[caption={PinMap Arrays}]
const PinMap PinMap_UART_TX[] = {
    {P0_19, UART_0, 1},
    {P1_13, UART_0, 3},
    {P1_27, UART_0, 2},
    { NC  , NC    , 0}
};

const PinMap PinMap_UART_RX[] = {
    {P0_18, UART_0, 1},
    {P1_14, UART_0, 3},
    {P1_26, UART_0, 2},
    {NC   , NC    , 0}
};
\end{lstlisting}

The reason for this is that the library is used to support a wide range of devices and configurations; the number of peripherals from use to use, and the way in which these are connected up by specific users, is entirely unpredictable, and leads to a more complex mapping structure being required. This is harder, and sometimes not possible, to achieve through the use of defined macros.

This not only results in wasted space, as the pins and their functions must be stored as part of the program code, but it also wastes valuable clock cycles, calculating values for pin addresses which could in theory be pre calculated. \verb|i2c_init()|, for example, first loops over the PinMap to find the memory address for the I2C with its SDA connected on pin 28, then repeats this process to search for the I2C with its SCL pin on port 27. Finally, it checks that these I2Cs are the same, before saving this memory address to a pointer for actual use. In total, this requires 5 function calls, and uses two loops to search over the data-mapping array; none of this can be optimised as it is spread across multiple compilation units. Setting just these two pin's mode and function requires a further 8 function calls, and 4 loops over the PinMap data structure.

It is plain to see that this is an extremely inefficient process and requires superfluous instructions to be added to the binary, especially as the device used for this project, the \verb|LPC11U24_401|, can have only one I2C bus and as such these functions can only ever return the same memory address each and every time they are run. In order to avoid this, these constant memory addresses were calculated by hand then saved in the form of a defined macro as used in the section above. The entirety of the pinmap data structure and supporting functions could then be removed from the project, drastically reducing its memory footprint.

\subsection{General Library Overhead}

A secondary side effect of using libraries is that they come with a large amount of general overhead; for example, the gyroscopic device that was used, the \verb|MPU6050| comes with a C++ class to interact with it. This, in turn, used its own secondary class, \verb|I2Cdev|, to interact with the device over the I2C bus. This class contained the logic to transform values into bit masks to allow the updating of individual bits within the device's configuration registers. These bit masks were passed on to the mBed library's \verb|I2C| class, which holds a \verb|struct| containing the pointer to the I2C memory address range, and acts as a wrapper for the mBed library's C I2C functions.

Defining, instantiating and storing a pointer, within a struct, within a class, within a class, within yet another class is an enormous overhead, especially as the middle layers of this abstraction are not adding significant functionality, but are merely acting as gatekeepers between the \verb|MPU6050| class and the lower level I2C functions. As part of the optimisation process, these unnecessary data structures were combined and removed where possible: pointers to memory addresses were replaced with definitions, as discussed above in section \ref{section:constant-memory-map}, bit masks were replaced with precomputed constants, and the MPU class was converted into a set of static functions to be called directly from \verb|main()|, removing the need to instantiate an object at all.
