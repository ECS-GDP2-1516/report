\namedsection{Tools Used}{Shepherd}

While regular meetings, both as a team and with the group's supervisors, are an essencial part of the project management process, it is also important to put in place a structure in which the members can communicate with each other throughout the week, and share work in an auditable fashion.

\subsection{Communication}
It was decided that the team should agree on a common platform on which to communicate with one another. As the majority of the team member run their own mail servers, and PGP keys which allow messages to be encrypted and signed, email was considered as a potentially secure and private option, especially considering some of ARM's System Verilog is under a non-public license. However, the group felt it was too unstructured to use and as such opted to use this only when large, one-off, files needed to be transferred between two people.

For general team conversations, the team decided to use Slack, which is a popular messaging tool among developers. It is similar to IRC in that it allows a team to create an arbitrary number of `channels' which group similar messages together such that multiple threads of conversations can be had simultaneously. However, it has a greater degree of centralisation, meaning all messages are stored and searchable, which was seen as an important requirement for structured group work. It also provides support for a variety of platforms, including web, Windows and most mobile systems, allowing the team members to maintain a point of contact throughout the day.

\subsection{Code Storage and Management}
In order to share code, the decision was made to use Git Version Control, a widely-used tool which imposes the requirement to `commit' atomic units of work in order for a developer to share their work. A commit log acts as an audit trail, tracking changes to text files and directory-like structures, to ensure that bugs and issues can be easily spotted and addressed. It is also very useful for collaboration, as it includes tools to perform `merges', when more than one developer has worked on a file at the same time; this is a task that would have to be performed manually without version control.

For this project, a range of repositories were created, all of which are publicly available on GitHub, a centralised website for storing Git repositories. The repositories were:

\begin{description}
    \item[ml-data] \hfill \\ The conversion scripts used to on movement data, as described in section \ref{sec:tools}, and the raw \verb|.csv| files of the raw data.
    \item[docs] \hfill \\ Administrative documents associated with the project: Agenda, Minutes, the Project Brief, Gantt Charts and the Ethical Approval Request.
    \item[ml-java] \hfill \\ Scaled down version of Weka, converted to a processing harness to output a neural network in optimised format.
    \item[ml-c] \hfill \\ An implementation of the algorithm, capable of running on a desktop machine.
    \item[mbed] \hfill \\ The implementation of the algorithm, with optimised mBed libraries, designed for the \verb|LPC11U24_401|.
    \item[report] \hfill \\ The repository for this report, including all diagrams and figures, and the underlying data used to generate the graphs in section \ref{ml}.
\end{description}

Git also keeps a full local copy of each repository on each developer's computer, allowing each member to continue their work even when not connected to the internet. This is acted as a backup, as well as a Git server which is run by Daniel Playle, which the group also pushed their changes to.

\subsection{Report Write-up}
Git proved successful at managing changes to the project's source code, and aiding in the administration overhead required when multiple people are working on the code simultaneously. Because of this, the group were keen to make use of Git for the report write-up too. This meant binary formats, such as those used by Microsoft Word and other WYSIWYG editors, were unacceptable; instead a text-based report write-up would be required, to allow Git to effectively hold it under source control.

\LaTeX is such a system; it uses one or more text files in which report content can be placed, and marked-up with the defined \TeX markup. This can be compiled into a professionally typeset PDF documents. The team found this solution very helpful as it allowed each member to focus solely on content creation rather than formatting, in the knowledge that the final report would be in a consistent and appropriate format. \LaTeX can also be used to create complex charts, tables and flow diagrams, not only reducing the time spent on such activities, but also ensuring a common `style' for all material produced by the group.
