\namedsection{Tools Used}{Shepherd}

While regular meetings, both as a team and with the group's supervisors, are an essencial part of the project management process, it is also important to put in place a structure in which the members can communicate with each other throughout the week, and share work in an auditable fashion.

As such, it was decided that the team should agree on a common platform on which to communicate with one another. Email was ruled out as the group felt it was too unstructured to use, and as such the team opted to use Slack, which allows users to send messages in a secure mannar. It allows a team to create an arbitrary number of `channels' which group similar messages together such that multiple threads of conversations can be had simultaniously. It is supported on a variety of platforms, including web, Windows and most mobile systems, allowing the team members to maintain a point of contact throughout the day.

In order to share code, the decision was made to use Git Version Control, a tool which allows users to track changes to text files and directory-like structures. This was very useful for this project, as it required development in both Java and C, allowing both sets of code to be kept in separate repositories, with all changes to the codebase being tracked in a logged an auditable manner.

As the decision to use Git had been made, the group also opted to use LaTeX to write the report and draw up the required figures and tables. This proved to be a good choice as LaTeX is also plaintext based, allowing it to be kept in its own Git repository too.