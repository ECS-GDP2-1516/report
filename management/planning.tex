\namedsection{Planning}{Shepherd}

Once the work had been split into individual units, as discussed above in section \ref{sec:division}, the team worked on the proposed Gantt charts for the project timeline. This required managing the work itself, but also setting aside time to work on the write up of the report, and preparation for the two Progress Seminars. It also had to take into account the University's timetable, so that other commitments such as exams and coursework could be avoided.

This section first details the proposed plans that were made at the beginning of the project (these can be viewed as Gantt charts in Appendix \ref{ch:gantt}) and then compares this with the realised progress and time management.

\subsection{Proposed Plan}

As the two progress seminars divided the term into roughly equal segments, it seemed sensible to set milestones for these to provide a framework around which to plan; this also ensured the content for each seminar was clearly defined in advance. As such, the group planned to complete the majority of background research and administrative work before the first seminar. This included applying for ethical approval, designing the hardware schematics, and ordering all the required components.

Directly after the progress seminar, the movement study would be conducted, allowing the software sub team to begin work on the development of the algorithm. During this, the hardware would be built, with the aim of having this completed two weeks prior to the second progress seminar, providing time to deploy the code onto the system and solve any problems which may arrive.

After the final progress seminar, the team planned to perform a second movement study to gauge the accuracy of the device, and perform any required changes in the weeks leading up to the Christmas break. This would allow the weeks during the holiday, and the time in term 2 to be spent focusing on writing the report.

\subsection{Realised Timeline}

As with all projects, external factors meant the team were not able to fully stick to the proposed project plans. The first delay was that the Research Governance Officer required further information about the dates surrounding the study and as such weren't able to approve the request immediately. As can be seen from Figure \ref{table:gantt:term11}, the movement study was placed early after the seminar, so there was an opportunity to make up the lost time despite the ethical approval delay forcing the movement study to be pushed back. However, this did delay the beginning of software development, as without the movement study results, the team were not able to accurately train a model.

A second unexpected factor is that it took the group longer to agree with the supervisors which platform should be used: the FPGA or the mBed. After a meeting with the customer, the decision was made to use both the FPGA and the mBed. This meant the work on the hardware development was not able to be completed prior to the second progress seminar, as the start of hardware work had been delayed slightly by the discussion and the decision resulted in the hardware team being effectively split in half, leaving fewer people to work on each platform.

Althought the team was able to recover from the two set backs detailed above, completing a working prototype in good time before the end of the project, the delays that these caused meant the group did not have time to complete the second movement study, and as such were not able to formally prove the effectiveness of the algorithm.
