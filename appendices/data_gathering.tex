\chapter{Participant Data Gathering}\label{app:participant}

Time period and place for gathering data:\newline
Wednesday: 18/11/2015 from 10:00-13:00.\newline
Thursday: 19/11/2015 from 10:00-13:00.\newline
Friday: 20/11/2015 from 11:00-14:00.\newline

This is taking place in Building 44, Room 4041.  

Before recording the data:
\begin{enumerate}
  \item The participants are read the participant document and if they agree, they sign the form.
  \item The participants are presented the exercises they need to reproduce according to \url{http://www.virgin-atlantic.com/gb/en/travel-information/your-health/inflight-exercise.html}. Participants will be presented a printout exercise list.
  \item Velcro bands are attached, being separated from the skin by cloth.
  \item The device (Samsung S3) containing an accelerometer and gyroscope is attached to the Velcro.
\end{enumerate}

How to record data:
\begin{enumerate}
  \item Open Physics Toolbox Suite Version 1.4.4 and open the Roller Coaster subsection.
  \item After attaching to the Velcro on the participant, the red + button is pressed in order to record.
  \item After the data is recorded for the required time period, the red STOP button is pressed to stop recording.
  \item The file is saved using a suggestive name for the specific exercise, having an index each participant (for example ``The Ballerina'' exercise for the 3rd participant will be named ballerina\textunderscore  3)
  \item Then, the Send via Gmail option will be selected and data will be sent to: gdp@dan.re.
\end{enumerate}

Additional indications:\newline
Right foot and right arm will be used to record data.
